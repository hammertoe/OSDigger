
\section{Introduction to the Business}
Etherworks, a startup IT venture was formed by four students from the University of Bristol, UK.  The main aim of the company was to develop and launch OSDigger, an archive/search service for mailing lists on the Internet.

The primary target is to archive mailing lists for Open Source Software development projects.  These projects are developed by loose-knit groups of individuals and organizations spread across the globe, whose primary means of co-ordination for these project is via mailing lists.

\subsection{What are Mailing Lists?}
Electronic mailing lists are similar to their conventional counterparts
in that they serve to distribute information to a subscribed audience.
With electronic mailing lists a particular email address is setup, eg.
\texttt{freebsd-questions@freebsd.org}.  Any email messages sent to that
address are automatically re-sent to all subscribers of the list.  There
may be thousands of subscribers on a list.  Anybody can subscribe to a
mailing list by sending a simple email to an automated management address
with the word \texttt{subscribe} in the body of the message.

\subsection{What is Open Source Software?}
The basic idea behind open source is very simple. When programmers on the Internet can read, redistribute, and modify the source for a piece of software, it evolves. People improve it, people adapt it, people fix bugs. And this can happen at a speed that, if one is used to the slow pace of conventional software development, seems astonishing.

Some facts about Open Source software:

%\begin{description}

%\item[Apache] The most popular web server on the Internet, it powers over
%60\% of the worlds web servers\footnote{The Netcraft Web Server Survey:
%http://www.netcraft.com/survey/}

%\item[Sendmail] The most widely used mail server software, routes mail on
%75 percent of the Internet's mail servers. \footnote{Sendmail Inc.:
%http://www2.sendmail.com/solutions/products/}

%\item[FreeBSD] A freely available Operating System.

%\item[GNU/Linux] An Operating System created by Finnish Student Linus
%Torvalds.  It has gained vast popularity over the last couple of years for
%use as  

%\item[Bind] The most widely used Domain Name Server on the Internet.  It
%is used to translate server addresses eg. www.osdigger.com into its
%numeric address so that data can be sent to it.

%\end{description}

\begin{itemize}
\item Linux-based OSes have the leading market share of Web servers
  powering the Internet's public Web sites, with 31 percent of all
  sites, according to a Netcraft study \cite{www:netcraft}.  The next most
  popular OS has only 20 percent.
  
\item Open-source e-mail routing tool Sendmail handles 80 percent of all
  the e-mail traffic on the net, according to a study by International Data Corp. (IDC).
  
\item More than 61 percent of all public Web sites are powered by the
  open-source Apache Web Server, according to a Netcraft study \cite{www:netcraft}.
  
\item DNS and Bind, the Domain Name Server tools that route all your URL
  requests across the Net, may represent as much as 100 percent market
  share, according to industry analysts.

\end{itemize}

{\bf In effect, the Internet itself would not function without the open-source tools that power it.}

\subsection{Why Open Source Software?}
As explained by Bob Young, founder of Red Hat Linux:

\begin{quote}
  The best analogy that illustrates this benefit is with the way we
  buy cars. Just ask the question, "Would you buy a car with the hood
  welded shut?" and we all answer an emphatic "No." So ask the
  follow-up question, "What do you know about modern
  internal-combustion engines?"  and the answer for most of us is,
  "Not much."
  
  We demand the ability to open the hood of our cars because it gives
  us, the consumer, control over the product we've bought and takes it
  away from the vendor.  We can take the car back to the dealer; if he
  does a good job, doesn't overcharge us and adds the features we
  need, we may keep taking it back to that dealer. But if he
  overcharges us, won't fix the problem we are having or refuses to
  install that musical horn we always wanted -- well, there are 10,000
  other car-repair companies that would be happy to have our business.
  
  In the proprietary software business, the customer has no control
  over the technology he is building his business around. If his
  vendor overcharges him, refuses to fix the bug that causes his
  system to crash or chooses not to introduce the feature that the
  customer needs, the customer has no choice. This lack of control
  results in high cost, low reliability and lots of frustration.
  
  With Open Source, companies like Red Hat are able to treat our
  customers as partners in the use of the technology they're building
  their businesses around.
  
  Having control over the technology they are using is the benefit
  that is enabling users of open-source tools to build more-reliable,
  more-customized and lower-cost systems than ever before.

\end{quote}

\subsection{Motivation}
The idea for the service arrived out of our own experiences in trying to
search through the vast back catalogue of mailing list messages to find
answers to questions we had that we knew had probably already been
addressed previously.

\section{Mission Statement} 

Arising from our own experiences, we intend to provide a service to allow the preservation and extraction of knowledge from mailing lists. We are aiming directly at the Open Source community to add value to the freely available mailing lists.  We also aim to help novice users to more easily enter the Open Source arena by allowing easy access to this knowledge.


\section{History and Position to Date}
So far a prototype site has been setup to demonstrate the workings of the
site.  The search engine, which is the main component of the site has been
mostly designed and written.  Users can currently enter a search query and
have the results returned to them.  More testing is needed before the site
is ready to be publically launched.

A bank account has been setup for the company.

The main developer, Matt Hamilton, is about to graduate from University
and will be working full-time on the final development of the site.

\section{Objectives, Near Term}
\begin{itemize}
\item To start an advertising campaign to 
\item To publically launch the site on the 1st of August
\end{itemize}

\section{Objectives, Long Term}
\begin{itemize}
\item To attract 200,000 visitors a month by the end of the first year of operation.
\item To develop the site into the most comprehensive source of Open Source information.
\end{itemize}

\section{Key Staff}
Etherworks is a partnership of four people whom met whilst studying at the University of Bristol, UK.  They all hold equal share of the company, and together personally funded \pounds2,000 for the purchase of a server for prototype work.  They all have a technical background, although several have experience of running their own small companies.  We are working with the Enterprise Center \cite{www:bec} at the University of Bristol, whom are providing the needed extra business and financial expertise.

\subsection{Matt Hamilton}
Matt, 22, is the main developer of the software behind the search engine.  The software was developed whilst studying a Computer Science (MEng) at the University of Bristol.  He graduates in June 2000 and intends to work full time on the final development of the OSDigger business.

He has run a part-time computer consulting business whilst at University with an annual turnover of \pounds8000.  His main contract, was installation and maintenance of an email system and virtual private network to link the Moscow and New York offices of Clintondale Aviation, a charter aviation company.  He also worked 4 months over summer 1999 with Hewlett Packard in Bristol, UK.

\subsection{Tim Saigol}
Tim, 22, currently works for IPL, a contract programming company in Bath, UK.  He graduated with a ii-1 in Computer Science (BSc Hons) in 1999.  Tim will be working part-time on the design and coding of various components of the search engine.

\subsection{Chris Green}
Chris, 22, currently works as a graphics designer for Netsight Internet Solutions, an Internet Service Provider in Bristol.  He graduated with a 1st in Computer Science (BSc Hons) in 1999.  He worked for IPL in Bath, UK for nine months before leaving to work with Netsight.  Chris will be working part-time on the graphics design of the web site.

\subsection{Chris Parsons}
Chris, 21, currently works for Elixer Studios a Games Programming company in London.  He ran a part-time web design business from University.  He graduated with a 1st in Computer Science (BSc Hons) in 1999.  Chris will be working part-time on the design of the web site layout and interaction with the search engine.

% LocalWords:  UK co eg freebsd org Netcraft http www netcraft com Sendmail IPL
% LocalWords:  sendmail Linux Torvalds osdigger publically MEng OSDigger Saigol
% LocalWords:  Clintondale BSc Hons Netsight Elixer
