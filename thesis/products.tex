\section{Product Description} 

The main product being developed by Etherworks is called OSDigger.  It is
an archive of email messages sent to Open Source Software mailing lists.  
The form of the archive will be a web site which users can use to
browse through the large back catalogue of messages and perform searches
to find specific topics.  The archive will be updated in real-time as
messages are sent out to the lists.  This means that the users will be
able to read current messages on the lists as easily as if they were
subscribed to them.

\subsection{Searching}
The main piece of software being developed is the search system, this is
the most complex piece and is the main product differentiator.  The search
will allow users to type in a query such as \emph{freebsd digital camera}
which could return messages about how to use a digital camera with FreeBSD
(an alternative operating system).

The search system has been custom written as there is not an existing
product on the market to fulfill the needs of the site.  We believe that
this gives us the competitive advantage.  As well as this, the search system incorporates some new features not yet seen in search engines generally.

It is estimated that the archive will hold several million messages, and
so the search system must be easy to use and quick.


\subsection{Browsing}
Users will also be able to browse through over 300 mailing lists, grouped by
subject.  The site will list the messages in chronological order, and
users can jump to a particular date should they want to.

It is intended that users will be able to use this feature to read the mailing list messages without having to subscribe themselves.  This is useful for people who just want to occasionally want to read what is being posted to the lists.  People who mainly read the lists, and do not post many messages will find this very useful.

Users will be able to access the list messages from anywhere using a standard web browser.  This paradigm of web-based messaging has been proved to be popular by sites such as HotMail, and Yahoo Mail.

\section{Readiness for Market}

The bulk of the search engine was written by Matt Hamilton as part of his
final year undergraduate project at the University of Bristol.  A
prototype of the site has been created, \texttt{http://beta.osdigger.com}, that demonstrates the main working points of the final commercial service.

\section{Applications}

The main application of the software is for the OSDigger web site.
However it is envisaged that there may be a market for the back end
software to be used for other purposes or maybe sold as a separate
product.

\section{Comparison of Competition}

There is one main competitor found in the market, GeoCrawler, whom are
described in Section \ref{geocrawler}.

\section{Legal Protection}
Software cannot be patented hence this is not an option.  Nor does it use
any new algorithms or processes which could be patented.  The software for
the search engine will be released under the GNU public license.  This is
an Open Source license that means that the software can be used for
whatever means and that the source code will always remain available.  
This is in
keeping with the subject of the lists being archived, and will provide the
Open Source community with a valuable resource.

\section{Product Differentiation}
As mentioned previously the main product differentiator is the custom
written search engine.  All of the other sites providing search facilities
use a generic search engine designed for indexing and searching web pages.
These are generally designed for small - medium sized web sites up to a
couple of thousand web pages.  As the archive will contain several million
messages, something more scalable is needed.  Another main problem is that
the archive is continually expanding, in the order of several thousand
 messages are
added each day.  Most already available search systems cannot
incrementally update their indexes and hence all of the messages must be
re-indexed.  This is not such a problem for a site with just the mailing
list archives for a handful of projects, but for such a large site, the
indexing would take too much time.

% LocalWords:  OSDigger freebsd http osdigger com GeoCrawler
